%%%%%%%%%%%%%%%%%%%%%%%%%%%%%%%%%%%%%%%%%
% Structured General Purpose Assignment
% LaTeX Template
%
% This template has been downloaded from:
% http://www.latextemplates.com
%
% Original author:
% Ted Pavlic (http://www.tedpavlic.com)
%
% Note:
% The \lipsum[#] commands throughout this template generate dummy text
% to fill the template out. These commands should all be removed when 
% writing assignment content.
%
%%%%%%%%%%%%%%%%%%%%%%%%%%%%%%%%%%%%%%%%%

%----------------------------------------------------------------------------------------
%	PACKAGES AND OTHER DOCUMENT CONFIGURATIONS
%----------------------------------------------------------------------------------------

\documentclass{article}

\usepackage{fancyhdr} % Required for custom headers
\usepackage{lastpage} % Required to determine the last page for the footer
\usepackage{extramarks} % Required for headers and footers
\usepackage{graphicx} % Required to insert images
\usepackage{lipsum} % Used for inserting dummy 'Lorem ipsum' text into the template

% Margins
\topmargin=-0.45in
\evensidemargin=0in
\oddsidemargin=0in
\textwidth=6.5in
\textheight=9.0in
\headsep=0.25in 

\linespread{1.1} % Line spacing

% Set up the header and footer
\pagestyle{fancy}
\lhead{\hmwkAuthorName} % Top left header
\chead{\hmwkClass\  : \hmwkTitle} % Top center header
\rhead{\firstxmark} % Top right header
\lfoot{\lastxmark} % Bottom left footer
\cfoot{} % Bottom center footer
\renewcommand\headrulewidth{0.4pt} % Size of the header rule
\renewcommand\footrulewidth{0.4pt} % Size of the footer rule

\setlength\parindent{0pt} % Removes all indentation from paragraphs

%----------------------------------------------------------------------------------------
%	DOCUMENT STRUCTURE COMMANDS
%	Skip this unless you know what you're doing
%----------------------------------------------------------------------------------------

% Header and footer for when a page split occurs within a problem environment
\newcommand{\enterProblemHeader}[1]{
\nobreak\extramarks{}{}\nobreak
\nobreak\extramarks{}\nobreak
}

% Header and footer for when a page split occurs between problem environments
\newcommand{\exitProblemHeader}[1]{
\nobreak\extramarks{}{}\nobreak
\nobreak\extramarks{}{}\nobreak
}

\setcounter{secnumdepth}{0} % Removes default section numbers
\newcounter{homeworkProblemCounter} % Creates a counter to keep track of the number of problems

\newcommand{\homeworkProblemName}{}
\newenvironment{homeworkProblem}[1][Problem \arabic{homeworkProblemCounter}]{ % Makes a new environment called homeworkProblem which takes 1 argument (custom name) but the default is "Problem #"
\stepcounter{homeworkProblemCounter} % Increase counter for number of problems
\renewcommand{\homeworkProblemName}{#1} % Assign \homeworkProblemName the name of the problem
\section{\homeworkProblemName} % Make a section in the document with the custom problem count
\enterProblemHeader{\homeworkProblemName} % Header and footer within the environment
}{
\exitProblemHeader{\homeworkProblemName} % Header and footer after the environment
}

\newcommand{\problemAnswer}[1]{ % Defines the problem answer command with the content as the only argument
\noindent\framebox[\columnwidth][c]{\begin{minipage}{0.98\columnwidth}#1\end{minipage}} % Makes the box around the problem answer and puts the content inside
}

\newcommand{\homeworkSectionName}{}
\newenvironment{homeworkSection}[1]{ % New environment for sections within homework problems, takes 1 argument - the name of the section
\renewcommand{\homeworkSectionName}{#1} % Assign \homeworkSectionName to the name of the section from the environment argument
\subsection{\homeworkSectionName} % Make a subsection with the custom name of the subsection
\enterProblemHeader{\homeworkProblemName\ [\homeworkSectionName]} % Header and footer within the environment
}{
\enterProblemHeader{\homeworkProblemName} % Header and footer after the environment
}
   
%----------------------------------------------------------------------------------------
%	NAME AND CLASS SECTION
%----------------------------------------------------------------------------------------

\newcommand{\hmwkTitle}{Project Proposal} % Assignment title
\newcommand{\hmwkClass}{ECSE\ 548} % Course/class
\newcommand{\hmwkAuthorName}{Group 2} % Your name

\begin{document}
\title{ECSE 548 (VLSI) Project Proposal - Direct-Mapped Cache}
\date{}
\author{Group 2}
\maketitle

\section{Progress on Achieving Goals indicated in Proposal}
- Microarchitecture : \textbf{Done} \\ \\
- Schematics of Cache (whole circuit and each component) :  \textbf{Done} \\ \\
- System Verilog Testbenches : \textbf{Partially Done} - Every component (\textbf{Mux, Demux, Comparator, SRAM Array}) except \textit{srams} were tested because \textit{srams} are analog and need to be tested using analog simulation software (SPICE in our case). SPICE simulations are in progress.
 \\ \\
- Schematics of Cache Controller : \textbf{Not Done} \\
	\textit{1.} Not enough time (midterms and paper submissions); \\
	 \textit{2.} We encountered design issues with Cache Controller (please refer to next section for more details)

\section{Problems Encountered So Far}

\textit{1.} Wrong assumption on the memory size of MIPS. - This was resolved after the clarifications from the Professor Meyer and the corresponding changes in design are represented in next section. \\ \\
\textit{2.} We discussed about two write policies: \textit{write-through} and \textit{write-back}. Due to controller complexity of write-back, we prefer to go for \textit{write-through}. \\ \\
\textit{3.} After read/write-miss, we have some issues  for the ouput at this particular time: usually  simple processors are supposed to wait. We need to discuss this with Professor.

\section{Modifications in Design}
\textit{1.} We corrected assumptions on memory size of MIPS to 256 Bytes. The sizes of Tag and Set signals were re-adjusted to 4-bit and 4-bit respectively according to the corrected assumption. \\ \\
\textit{2.} For the performance evaluation, SPICE simulation is a must instead of a potential option. \\ \\
\textit{3.} Integration to MIPS becomes a must instead of a potential option. \\ \\


\end{document}